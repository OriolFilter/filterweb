\subsection{System requirements}\label{subsec:system-requirements}
\begin{flushleft}
    To install and configure the packages, a user with root access might be required, in case your system doesn't have the package
    "\textbf{sudo}", and your user be configured with it, consider swapping to the Root user.
\end{flushleft}

\begin{itemize}
    \item Access to internet
    \item Git
    \item Docker
    \item Docker-compose
\end{itemize}

\subsubsection{Git installation}
\paragraph{apt}
\begin{flushleft}
\begin{lstlisting}[language=bash,label={lst:apt-git}]
sudo apt-get update && sudo apt-get install git
\end{lstlisting}
\end{flushleft}

\paragraph{pacman}
\begin{flushleft}
\begin{lstlisting}[language=bash,label={lst:pacman-git}]
sudo pacman -Syu && sudo pacman -S git
\end{lstlisting}
\end{flushleft}

\paragraph{apk}
\begin{flushleft}
\begin{lstlisting}[language=bash,label={lst:apk-git}]
sudo apk update && sudo apk add --no-cache git
\end{lstlisting}
\end{flushleft}


\subsubsection{Docker installation}

\paragraph{apt}
\begin{flushleft}
\begin{lstlisting}[language=bash,label={lst:apt-docker}]
sudo apt-get update && sudo apt-get install docker-ce docker-ce-cli containerd.io
\end{lstlisting}
\end{flushleft}

\paragraph{pacman}
\begin{flushleft}
\begin{lstlisting}[language=bash,label={lst:pacman-docker}]
sudo pacman -Ss && sudo pacman -S docker
\end{lstlisting}
\end{flushleft}

\paragraph{apk}
\begin{flushleft}
\begin{lstlisting}[language=bash,label={lst:apk-docker}]
sudo apk update && sudo  apk add --no-cache docker
\end{lstlisting}
\end{flushleft}

\subsubsection{Docker configuration - allow user to use docker}
\begin{lstlisting}[language=bash,label={lst:add-group-docker}]
sudo usermod -a -G docker your_user
\end{lstlisting}

\subsubsection{Docker configuration - enable docker on boot}
\paragraph{service}
\begin{flushleft}
\begin{lstlisting}[language=bash,label={lst:service-docker}]
sudo service enable docker
\end{lstlisting}
\end{flushleft}

\paragraph{systemctl}

\begin{flushleft}
\begin{lstlisting}[language=bash,label={lst:systemctl-docker}]
sudo systemctl start
\end{lstlisting}
\end{flushleft}

\paragraph{rc-update}
\begin{flushleft}
\begin{lstlisting}[language=bash,label={lst:rc-docker}]
sudo run rc-update add docker boot
\end{lstlisting}
\end{flushleft}

\subsubsection{Docker-Compose installation}
\paragraph{apt}
\begin{flushleft}
\begin{lstlisting}[language=bash,label={lst:apt-compose}]
sudo apt-get update && sudo apt-get install python3
sudo curl -L "https://github.com/docker/compose/releases/download/1.29.2/docker-compose-$(uname -s)-$(uname -m)" -o /usr/local/bin/docker-compose
sudo chmod +x /usr/local/bin/docker-compose
\end{lstlisting}
\end{flushleft}

\paragraph{pacman}
\begin{flushleft}
\begin{lstlisting}[language=bash,label={lst:pacman-compose}]
sudo pacman -Ss && sudo pacman -S python3
sudo curl -L "https://github.com/docker/compose/releases/download/1.29.2/docker-compose-$(uname -s)-$(uname -m)" -o /usr/local/bin/docker-compose
sudo chmod +x /usr/local/bin/docker-compose
\end{lstlisting}
\end{flushleft}

\paragraph{apk}
\begin{flushleft}
\begin{lstlisting}[language=bash,label={lst:apk-compose}]
sudo apk update && sudo  apk add --no-cache py-pip python3-dev libffi-dev openssl-dev gcc libc-dev rust cargo make
sudo curl -L "https://github.com/docker/compose/releases/download/1.29.2/docker-compose-$(uname -s)-$(uname -m)" -o /usr/local/bin/docker-compose
sudo chmod +x /usr/local/bin/docker-compose
\end{lstlisting}
\end{flushleft}


\subsubsection{Repository installation}\label{subsubsec:repo-installation}
\begin{flushleft}
    The only step is to clone the repository.
    \begin{flushleft}
        \textbf{git clone \url{\repoURL}}
    \end{flushleft}
\end{flushleft}




\subsection{Repository deployment minimal customization}\label{subsec:repo-customization}
\subsubsection[Main server deployment minimal customization]{Main server deployment minimal customization}
\lstinputlisting[language=bash, label={lst:main_env_file}]{../.env}
\begin{flushleft}
    The only to edit it's the \textbf{hostname}, since we need to replace it for our public/local \textbf{IP}
    or our \textbf{Domain Name}, in case we just wanted to test it, we could use "\textbf{localhost}".
\end{flushleft}

\newpage
\subsubsection[Backups client deployment minimal customization]{Backups client deployment minimal customization}
\begin{flushleft}
    \lstinputlisting[language=bash,label={lst:env_bk_pg}]{../bkcli_env_folder/.env_bk_pg}
    \lstinputlisting[language=bash,label={lst:env_bk_nx}]{../bkcli_env_folder/.env_bk_nx_logs}
    First we need to specify the \textbf{SFTHOST} for the \textbf{SFTP} server address, either the \textbf{IP} or the
    \textbf{Domain Name}, in this case, we can't use "localhost", in case we hosted the \textbf{SFTP} server in the same machine
    than the \textbf{bakcup\_dealer}, we need to specify our local \textbf{IP}.
\end{flushleft}

\subsubsection[Cron periodical backups minimal customization]{Cron periodical backups minimal customization}
\begin{flushleft}
    Once we have modified both \textbf{.env\_bk\_pg} and \textbf{.env\_bk.\_nx\_logs}, we need to execute the next command in order to do
    a backup every day.
    \begin{lstlisting}[language=bash,label={lst:insert_to_cron}]
echo "* * */1 * * $USER docker-compose -f $(pwd)/backup_dealer-compose.yml --env-file $(pwd)/bkcli_env_folder/.env_bk_pg up" | sudo tee -a  /etc/cron.d/docker_backups
echo "* * */1 * * $USER docker-compose -f $(pwd)/backup_dealer-compose.yml --env-file $(pwd)/bkcli_env_folder/.env_bk_ng_logs up"  | sudo tee -a  /etc/cron.d/docker_backups
    \end{lstlisting}
\end{flushleft}

\newpage
\subsubsection[Backups Server deployment minimal customization]{Backups Server deployment minimal customization}
\begin{flushleft}
    In case that we desire to host the server \textbf{SFTP} during the testing, we can skip this step.
\end{flushleft}

\begin{flushleft}
    The next step is copy the folder "backup\_server" to the device that we want to use as a backup server, this step
    isn't necessary if we already have a \textbf{SFTP} server, or we desire to do local backups(which isn't recommendable).

    To copy the folder to a remote server we can use the command:
\begin{lstlisting}[language=bash,label={lst:scp}]
scp -r ./backup_server <user>@<host>:~
\end{lstlisting}
\end{flushleft}

\subsection{Repository deployment booting services}\label{subsec:repository-deployment-booting-services}
\subsubsection[Main Server Service Booting]{Main Server Service Booting}
This can simply be done by executing the next command:
\begin{lstlisting}[language=bash,label={lst:compose-up}]
docker-compose build && docker-compose up
\end{lstlisting}

\subsubsection[Backup (Remote) Server Service Booting]{Backup (Remote) Server Service Booting}
\textit{Reminder that the server needs to fulfill the requirements.}
This can simply be done by executing the next command:
\begin{lstlisting}[language=bash,label={lst:compose-up-bk-remote}]
ssh <user>@<host> docker-compose -f backup_server/docker-compose.yml up -d
\end{lstlisting}

\subsubsection[Backup (Local) Server Service Booting]{Backup (Local) Server Service Booting}
This can simply be done by executing the next command:
\begin{lstlisting}[language=bash,label={lst:compose-up-bk-Local}]
docker-compose -f backup_server/docker-compose.yml up -d
\end{lstlisting}


\subsubsection[Use your own email sender]{Use your own email sender}
\begin{flushleft}
    To change the current email sender, we must go to the file "\textbf{mailer.php}", situated at "\textit{./config/filterweb/private/libraries}".
    Once we are in the file, we just need to change the next values to the desired ones:
    \textit{If you are using a Gmail account, access from untrusted applications needs to be enabled}
    \begin{lstlisting}[language=php,label={lst:mail_accounts}]
        $mail_server='filter.web.asix@gmail.com';
        $mail_server_pass='ITB2019015';
    \end{lstlisting}
    In case you don't use gmail, you might need to modify:
    \begin{lstlisting}[language=php,label={lst:mail_accounts2}]
        $mail->Host       = 'smtp.gmail.com';
        $mail->Port       = 587;
    \end{lstlisting}
    And finally, to not keep being sending messages as "Arcade Shop", replace this with ur desired name:
    \begin{lstlisting}[language=php,label={lst:mail_accounts3}]
        $mail->setFrom($mail_server, 'Arcade Shop');
    \end{lstlisting}
\end{flushleft}


\newpage
\subsection{Repository deployment further customization}\label{subsec:repository-deployment-further-customization}
\subsubsection[Custom SSL Keys]{Custom SSL Keys}
\begin{flushleft}
    In order to change the current certificates, you need to replace them for yours, the location of each certificate
    it's in the \textbf{docker-compose} document

    This applies for the \textbf{NGINX} and \textbf{SFTP} containers.
\end{flushleft}

\subsubsection[Custom Keychains]{Custom Keychains}
\paragraph{Requirements:} Keychain package installed.
\begin{flushleft}

    Once the requirements are fulfilled.
    We can proceed to generate the keychains.
    \begin{lstlisting}[label={lst:generating_keychain,language=bash}]
ssh-keygen -t rsa -b 2048 -f ./id_rsa
    \end{lstlisting}
\end{flushleft}
\begin{flushleft}
    Once we have created the keys, it's time to replace them, remember that the files need to be replaced in the local
    system AND the server in case the SFTP server is remote.
\end{flushleft}
\begin{flushleft}
    In case of modifying the path to the keys in the docker-compose file, remember that the \textbf{backup dealer} also
    makes use of these keys in order to login without need of password, so in case of changing its path, you need to do
    it in both files.
\end{flushleft}
%

\subsubsection[Add folders to the SFTP Service]{Add folders to the SFTP Service}
\begin{flushleft}
    In order to do this, we need to edit the file "\textit{users.conf}", located in the same folder as the \textbf{SFTP} server.
    \lstinputlisting[language=bash,label={lst:users.conf}]{../backup_server/users.conf}
    The file structure is:

\begin{lstlisting}[language=bash,label={lst:users.conf_hash_structure}]
USER:hashed_password:e:UID:GUID:folder_list
\end{lstlisting}
    In our case, since we don't need the user to have a password, since we are using keychains, we can just skip that field.
    \begin{lstlisting}[language=bash,label={lst:users.conf_structure}]
USER::UID:GUID:folder_list
    \end{lstlisting}
\end{flushleft}

\subsubsection[Deploying Custom Databases]{Deploying Custom Databases}
\begin{flushleft}
    In order to use/create our custom databases, we have 2 options, using a volume or directory that already has them
    created, or using the script that triggers if the directory "\textit{/data}" is empty, happen on mounting an empty
    volume, or the first time the container starts.
\end{flushleft}
\paragraph{Script/Building explanation}
\begin{flushleft}
    This lines determine which files will search for:
    \begin{lstlisting}[language=bash,label={lst:declare_array}]
declare -a FORMAT_ARR=('_skel' '_users' '_val')
declare -a DATABASE_ARR=( $( awk '{ gsub(","," "); gsub("  "," "); gsub(" ","\n"); print}' <<< "$BUILD_DATABASE_LIST" ) );
    \end{lstlisting}
    First we have an array declared with 3 values, this are the suffix that will be looked for when searching the SQL files.
    In the second line, we receive the variable \textbf{BUILD\_DATABASE\_LIST}, which will be normalized, first, comas transform
    in spaces,afterwards, double spaces transformed in single spaces, and finally the spaces left will be transformed in newlines,
    which will be used to create new entries in the array.

    \begin{flushleft}
        Once the arrays are generated, the script will attempt to generate a database for every entry in the
        \textbf{DATABASE\_ARR}, keep in mind that it's the values given by the user from the docker-compose configuration.
    \end{flushleft}

    \begin{lstlisting}[language=bash,label={lst:for_array}]
for key in "${!FORMAT_ARR[@]}"
do
    sqlfile="${WORKDIR}/${database}${FORMAT_ARR[$key]}.sql"
    \end{lstlisting}
    Basically, will try to find an element that match the name \textbf{GIVEN\_DATABASE\_NAME}+\textbf{SUFFIX} from the array,
    which means that for every entry in the \textbf{DATABASE\_ARR}, will search a "\textbf{DATABASE\_ARR}\_skel",
    "\textbf{DATABASE\_ARR}\_users", "\textbf{DATABASE\_ARR}\_val", executed in this order, that's very important to
    keep in mind in case we make use of multiple SQL files, since the order of their execute might alter the result.
\end{flushleft}

\paragraph{Configuration}
\begin{flushleft}
    The first step is to change the "\textbf{.env}" file with the desired database names, keep mind that the elements
    can se separated by spaces or comas.
    \lstinputlisting[language=bash,label={lst:.env}]{../.env}
    Once the database names been renamed, it's time to replace the current \textbf{SQL} files in the folder
    \textit{Dockerfiles/postgresql/sources}.
\end{flushleft}

\begin{flushleft}
    For example, if we want to create the database "new\_database", the files must be named "new\_database\_skel.sql",
    "new\_database\_users.sql", "new\_database\_val.sql".
\end{flushleft}
