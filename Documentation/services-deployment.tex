\subsection{System requirements}\label{subsec:system-requirements}
\begin{flushleft}
    To install and configure the packages, a user with root access might be required, in case your system doesn't have the package
    "\textbf{sudo}", and your user be configured with it, consider swapping to the Root user.
\end{flushleft}

\begin{itemize}
    \item Access to internet
    \item Git
    \item Docker
    \item Docker-compose
\end{itemize}

\subsubsection{Git installation}
\paragraph{apt}
\begin{flushleft}
\begin{lstlisting}[language=bash,label={lst:apt-git}]
sudo apt-get update && sudo apt-get install git
\end{lstlisting}
\end{flushleft}

\paragraph{pacman}
\begin{flushleft}
\begin{lstlisting}[language=bash,label={lst:pacman-git}]
sudo pacman -Syu && sudo pacman -S git
\end{lstlisting}
\end{flushleft}

\paragraph{apk}
\begin{flushleft}
\begin{lstlisting}[language=bash,label={lst:apk-git}]
sudo apk update && sudo apk add --no-cache git
\end{lstlisting}
\end{flushleft}


\subsubsection{Docker installation}

\paragraph{apt}
\begin{flushleft}
\begin{lstlisting}[language=bash,label={lst:apt-docker}]
sudo apt-get update && sudo apt-get install docker-ce docker-ce-cli containerd.io
\end{lstlisting}
\end{flushleft}

\paragraph{pacman}
\begin{flushleft}
\begin{lstlisting}[language=bash,label={lst:pacman-docker}]
sudo pacman -Ss && sudo pacman -S docker
\end{lstlisting}
\end{flushleft}

\paragraph{apk}
\begin{flushleft}
\begin{lstlisting}[language=bash,label={lst:apk-docker}]
sudo apk update && sudo  apk add --no-cache docker
\end{lstlisting}
\end{flushleft}

\subsubsection{Docker configuration - allow user to use docker}
\begin{lstlisting}[language=bash,label={lst:add-group-docker}]
sudo usermod -a -G docker your_user
\end{lstlisting}

\subsubsection{Docker configuration - enable docker on boot}
\paragraph{service}
\begin{flushleft}
\begin{lstlisting}[language=bash,label={lst:service-docker}]
sudo service enable docker
\end{lstlisting}
\end{flushleft}

\paragraph{systemctl}
\begin{lstlisting}[language=bash,label={lst:systemctl-docker}]
sudo systemctl start
\end{lstlisting}

\paragraph{rc-update}
\begin{flushleft}
\begin{lstlisting}[language=bash,label={lst:rc-docker}]
sudo run rc-update add docker boot
\end{lstlisting}
\end{flushleft}

\subsubsection{Docker-Compose installation}
\paragraph{apt}
\begin{flushleft}
\begin{lstlisting}[language=bash,label={lst:apt-compose}]
sudo apt-get update && sudo apt-get install python3
sudo curl -L "https://github.com/docker/compose/releases/download/1.29.2/docker-compose-$(uname -s)-$(uname -m)" -o /usr/local/bin/docker-compose
sudo chmod +x /usr/local/bin/docker-compose
\end{lstlisting}
\end{flushleft}

\paragraph{pacman}
\begin{flushleft}
\begin{lstlisting}[language=bash,label={lst:pacman-compose}]
sudo pacman -Ss && sudo pacman -S python3
sudo curl -L "https://github.com/docker/compose/releases/download/1.29.2/docker-compose-$(uname -s)-$(uname -m)" -o /usr/local/bin/docker-compose
sudo chmod +x /usr/local/bin/docker-compose
\end{lstlisting}
\end{flushleft}

\paragraph{apk}
\begin{flushleft}
\begin{lstlisting}[language=bash,label={lst:apk-compose}]
sudo apk update && sudo  apk add --no-cache py-pip python3-dev libffi-dev openssl-dev gcc libc-dev rust cargo make
sudo curl -L "https://github.com/docker/compose/releases/download/1.29.2/docker-compose-$(uname -s)-$(uname -m)" -o /usr/local/bin/docker-compose
sudo chmod +x /usr/local/bin/docker-compose
\end{lstlisting}
\end{flushleft}


\subsubsection{Repository installation}\label{subsubsec:repo-installation}
\begin{flushleft}
    The only step is to clone the repository.
    \begin{flushleft}
        \textbf{git clone \url{\repoURL}}
    \end{flushleft}
\end{flushleft}




\subsection{Repository customization}\label{subsec:repo-customization}
\subsubsection[Main server customization]{Main server customization}
\lstinputlisting[language=bash, label={lst:main_env_file}]{../.env}